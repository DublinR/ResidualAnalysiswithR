\documentclass[residuals.tex]{subfiles}

\begin{document}

\Large
\section{Assumption of Constant Variance}
\subsection*{Homoscedasticity}
\begin{itemize}
\item \textbf{\textit{Homoscedascity}} is the technical term to describe the variance of the residuals being constant across the range of predicted values. 

\item \textbf{\textit{Heteroscedascity}} is the converse scenario : the variance differs along the range of values.
\end{itemize}


\noindent Suppose you plot the individual residuals against the predicted value, the variance of the residuals predicted value should be constant. 
\bigskip
\noindent Consider the red arrows in the picture below, intended to indicate the variance of the residuals at that part of the number line. For the OLS summption to be valid , the length of the red lines should be more or less the same.

\begin{figure}[h!]
\centering
\includegraphics[width=0.6\linewidth]{homosked}
\caption{}
\label{fig:homosked}
\end{figure}


\newpage
\begin{figure}[h!]
\centering
\includegraphics[width=0.7\linewidth]{homosked2.png}
\caption{}
\label{fig:homosked2}
\end{figure}
\newpage

%http://stats.stackexchange.com/questions/58141/interpreting-plot-lm
%\texttt{I explained the assumption of homoscedasticity and the plots that can help you assess it (including scale-location plots [2]) on CV here: What does having constant variance in a linear regression model mean? I have discussed qq-plots [3] on CV here: QQ plot does not match histogram. So, what's left is primarily just understanding [5], the residual-leverage plot.}

\end{document}
