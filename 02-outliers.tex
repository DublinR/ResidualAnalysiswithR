Introduction
Let's begin our discussion on robust regression with some terms in linear regression.

Residual: The difference between the predicted value (based on the regression equation) and the actual, observed value.

Outlier: In linear regression, an outlier is an observation with large residual. In other words, it is an observation whose dependent-variable value is unusual given its value on the predictor variables. An outlier may indicate a sample peculiarity or may indicate a data entry error or other problem.

Leverage: An observation with an extreme value on a predictor variable is a point with high leverage. Leverage is a measure of how far an independent variable deviates from its mean. High leverage points can have a great amount of effect on the estimate of regression coefficients.

Influence: An observation is said to be influential if removing the observation substantially changes the estimate of the regression coefficients.  Influence can be thought of as the product of leverage and outlierness.

Cook's distance (or Cook's D): A measure that combines the information of leverage and residual of the observation.


% - http://stats.stackexchange.com/questions/65912/precise-meaning-of-and-comparison-between-influential-point-high-leverage-point


\begin{quote}
"Outliers are sample values that cause surprise in relation to the majority of the sample" (W.N. Venables and B.D. Ripley. 2002. Modern applied statistics with S. New York: Springer, p.119).
\end{quote}

Crucially, surprise is in the mind of the beholder and is dependent on some tacit or explicit model of the data. There may be another model under which the outlier is not surprising at all, say if the data really are lognormal or gamma rather than normal.

\end{document}
